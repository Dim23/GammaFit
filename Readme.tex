\documentclass[11pt]{article}
\usepackage[utf8]{inputenc}
\usepackage{xcolor}
\usepackage{listings}
\lstset{basicstyle=\ttfamily,
  showstringspaces=false,
  commentstyle=\color{red},
  keywordstyle=\color{blue}
}
\usepackage{hyperref}
\urlstyle{same}
\usepackage{geometry}
 \geometry{
 a4paper,
 total={170mm,257mm},
 left=20mm,
 top=20mm,
 }

\title{Quick tutorial for centrality determination using direct impact parameter reconstruction in MPD (NICA)}
\author{Dim Idrisov, Petr Parfenov, Vinh Ba Luong and Arkadiy Taranenko}
\date{March 2021}

\begin{document}

\maketitle

\section{The direct impact parameter reconstruction}

\subsection{Installation}
Below is a detailed description of the program used to reconstruct the distribution of the impact parameter.
To start the fitting process, dowland script from \url{https://github.com/Dim23/GammaFit.git}\\.

\subsection{Fitting model data}
In the case of fitting model data, it is sufficient to have only a histogram with a multiplicity distribution. Use following commands to run \texttt{GammaFit.C} :

\begin{lstlisting}[language=bash,caption={}]
 root -b GammaFit.C(fileadres, current_mult, outadres, minNch)
\end{lstlisting}
Where the arguments are:
\begin{itemize}
\item \texttt{fileadres} - input root file.
\item \texttt{current\_mult} - Histogram of multiplicity.
\item \texttt{outadres} - output file from fitting process.
\item \texttt{minNch} - lower value of the fitting area ( `20` by default).
\end{itemize}

to set the value of the cross section in line 17 set the desired value of the variable \texttt{sigma}.

\subsection{Fitting reconstracted data}
When fitting the reconstructed data, it is necessary to take into account the efficiency of the detectors. To take these features into account, normalization is made to non-reconstructed model data.

To start the fitting process, run \texttt{GammaFit.C} with next option:
\begin{lstlisting}[language=bash,caption={}]
 root -b GammaFit.C(fileadres, current_mult, outadres, minNch, efficiencyFit,fileadres2, current_mult2)
\end{lstlisting}
Where the arguments are:

\begin{itemize}
\item \texttt{fileadres} - input root file with multiplisity of the reconstructed data.
\item \texttt{current\_mult} - Histogram of multiplicity.
\item \texttt{outadres} - output file from fitting process.
\item \texttt{minNch} - lower value of the fitting area ( `20` by default).
\item \texttt{efficiencyFit} - should be set `true` ( `false` by default).
\item \texttt{fileadres2} - input root file with multiplisity of non-reconstructed model data.
\item \texttt{current\_mult2} - Histogram with multiplisity of non-reconstructed model data.
\end{itemize}
\section{OUTPUT}
Resulting file \texttt{outadres} will contain TCanvas with fit results and data-to-fit ratio - \texttt{Canvas\_0f\_fit\_result}.
Where the - \texttt{fit\_func} is the resulting fit function of the multiplicity distribution .
\texttt{FitResult} - TTree with fit parameters of fit function.
\texttt{Result} - TTree with min and max percent of centrality and also the boundaries of the centrality classes.

\texttt{TTree} contains the following information about each centrality class:
\begin{itemize}
    \item \texttt{MinPercent} -- minimum value of centrality in the given centrality class
    \item \texttt{MaxPercent} -- maximum value of centrality in the given centrality class
    \item \texttt{MinBorder} -- lower cut on multiplicity for the given centrality class
    \item \texttt{MaxBorder} -- upper cut on multiplicity for the given centrality class.
\end{itemize}

\texttt{fit\_B\_Mean} - TGraphErrors of impact parametr as a function of centrality .

\texttt{ImpactParametDist\_CENT*\_*} - histograms with the distribution of the impact parameter in centrality class.
\subsection{Using centrality classes provided from the framework in the analysis}
The file \texttt{outadres} have all needed information about centrality class.
Use macro \texttt{printFinal.C} to display this information in a simple and readable way:
\begin{lstlisting}[language=bash,caption={}]
 root -l -b -q printFinal.C'("path-to-FINAL.root")
\end{lstlisting}
This will print out all needed information for each centrality class.
This macro also can save output information in latex and csv tables format.
Example of \texttt{printFinal.C} saving in latex table:
\begin{lstlisting}[language=bash,caption={}]
 root -l -b -q printFinal.C'("path-to-FINAL.root","./example.tex")
 \end{lstlisting}
 Example of \texttt{printFinal.C} saving in csv table (compatible with LibreOffice and MS Excel):
\begin{lstlisting}[language=bash,caption={}]
 root -l -b -q printFinal.C'("path-to-FINAL.root","./example.csv")
\end{lstlisting}
Example of \texttt{printFinal.C} saving in C++ code:
\begin{lstlisting}[language=bash,caption={}]
root -l -b -q printFinal.C'("path-to-FINAL.root","./example.C")'
\end{lstlisting}
%
After \texttt{printFinal.C} generates output C++ code, one can use \texttt{Float\_t GetCentMult(Int\_t)} as a function which returns centrality percent based on input multiplicity value.
\end{document}
